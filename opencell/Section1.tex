\section{Introduction}
Video streaming has been a dominant application traffic in cellular networks, it consists
up to 40\% of the peak load~\cite{LTENetwork, VideoMeasureatt}. Although
the video streaming has become one huge part of cellular networks, most video streaming algorithms are designed only for wireline networks~\cite{BBA, QDASH, Festive}. Unique characteristics from the cellular networks, such as bandwidth un-predictability over short period of
time due to handoff or fading effects, create a challenge for a satisfactory
video streaming service. 

Most of the current video streaming services are using DASH (\textbf{D}ynamic
\textbf{A}daptation \textbf{S}treaming over \textbf{H}TTP)~\cite{DASH} based
end-to-end approaches. It uses chunk-based downloading and adjusts its video
streaming bit rate based on the play progress and the monitored bandwidth. The
application stitches video chunks together to provide a continuous playback.
However most of the video streaming service are not tailored for cellular networks, and they suffer from a conservative bandwidth estimation and video quality oscillation due to a lack of knowledge for the network condition. 
Some low layer connection information such as signal information is hard to access from application layer without root access.
End-to-end mechanisms~\cite{Sprout, QDASH, Festive} have been proposed to estimate link throughput, however some aims to reduce the packet delay while trading off the throughput, while others are making conservative and rough estimation. Periodical downloading behavior~\cite{OnOff} makes the bandwidth prediction even harder.
%For TCP throughput, the application need constantly saturate the link for a better estimation and the estimation takes a long time to converge, which can be hardly used in video streaming case where downloading is periodical\cite{OnOff,Sprout}. 
%\xin{Not sure if this is correct. Can they be collected from end
%hosts? Cite some papers or explain why it's hard}, \kelvin{ how does this sound like now?}

On the other hand, a cellular network service provider has much better knowledge of the connection
conditions for the clients via monitoring, as the last mile has long been identified as the
bottleneck\cite{LASTMILE}. Cellular ISPs also know the number of
active users in each cell and proportional fair packet scheduling system, and thus can make more
precise predictions in terms of available bandwidth. In fact, many key statistics like radio information and the base station utilization can be easily and some are
already being collected within carriers. 

Given cellular carrier's advantages, there are some proposals\cite{Avis} for an in-network video streaming framework where the network controls video streaming algorithm. 
Nevertheless it also suffers from some limitations. First it
lacks the application knowledge such as client video buffer occupancy and may lead to re-buffering. 
Second different users may apply different subscription
policies w.r.t.video quality (Spotify already has low quality and high quality for free and premium users, video service may apply the same concept) and streaming service providers may
hesitate to share user information with ISPs. Users may also have data budget concerns. Last
but not least, in-network approaches may require the modification of packet scheduling system and this
is hard in practice as most carriers are using pre-installed
vendor-specific scheduler. 

To address the issue, we propose a joint optimization between the cellular network (ISP) and video streaming service (CP): ISP exposes its estimated bandwidth to CP, and CP adjusts its streaming service decision based on the estimated bandwidth. By doing so video streaming service can achieve higher quality and reduce the amount of chunk prefetching, and thus save data budget and power consumption. ISPs also benefit from less bursty traffic and avoid excessive queueing delay from flow pacing. There are three main contributions in this paper:
\begin{itemize}
\item Design a scalable architecture to compute  and expose them to video streaming providers on the fly.
\item Conduct a measurement study to show the predictability of user channel quality and cell loads. 
\item Design an algorithm to show that by using the KPIs we have a significant improvement in video streaming service.
\end{itemize}

This paper introduces a scalable architecture to compute the bandwidth and expose the KPIs to the service provider in \autoref{sec:Architecture} and then we conduct a preliminary measurement study of predictability of available bandwidth in \autoref{sec:prediction}. We propose our new optimization algorithm in \autoref{sec:optimization} and show our results in \autoref{sec:evaluation}. This paper concludes in \autoref{sec:conclusion}. 


