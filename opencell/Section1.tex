\section{Introduction}
Video streaming has been the dominant factor in cellular networks, it consists
up to 40\% of the traffic peak load~\cite{LTENetwork, VideoMeasureatt}. Although
the video streaming has become one big part of cellular networks, the
performance for video streaming is not promising. Unique characteristics from
the cellular networks, such as bandwidth un-predictability over short period of
time due to handoff or fading effects, create a challenge for a satisfactory
video streaming service. 

Most of the current video streaming services are using DASH (\textbf{D}ynamic
\textbf{A}daptation \textbf{S}treaming over \textbf{H}TTP)\cite{DASH} based
end-to-end approaches. It uses chunk-based downloading and adjusts its video
streaming bit rate based on the play progress and the monitored bandwidth. The
application stitches video chunks together to provide a continuous playback.
However most of the video streaming service are not tailored for cellular networks, and they suffer from a conservative estimation and oscillation\cite{BBA,Festive, QDASH} .The reason for this is the lack of knowledge for the network condition. 
Some low layer connection information such as signal information is hard to access from application layer without root access.
Some mechanisms like Sprout\cite{Sprout} have been proposed to estimate link throughput, however it aims to reduce the packet delay while trading off the throughput, and need continuously saturate the link, which can be hardly used in video streaming case where downloading is periodical\cite{OnOff}.   
%For TCP throughput, the application need constantly saturate the link for a better estimation and the estimation takes a long time to converge, which can be hardly used in video streaming case where downloading is periodical\cite{OnOff,Sprout}. 
\xin{Not sure if this is correct. Can they be collected from end
hosts? Cite some papers or explain why it's hard}, \kelvin{ how is this sounds like now?}

On the other hand, a cellular network service provider has much better knowledge of the connection
conditions for the clients via monitoring, as the last mile has long been identified as the
bottleneck\cite{LASTMILE}. Cellular ISPs also know the number of
active users in each cell and the scheduling system, and thus can make more
precise predictions in terms of available bandwidth. In fact, many key statistics can be easily and some are
already being collected within carriers. 

Given cellular carrier's advantages, AVIS\cite{Avis} proposes an in-network video streaming framework where the network makes decision. 
Nevertheless it also suffers from some constraints. First it
lacks the application knowledge of play progress and may lead to re-buffering. 
Second different users may have different subscription
policies w.r.t. video streaming service and streaming service providers may
hesitate to share it with ISPs. Users may also have data budget concerns. Last
but not least, it may require the modification of the cellular scheduling system and that
may not be practical since most are using pre-installed
vendor-specific scheduler. 

To address the issue, we propose a novel cellular network design that exposes our in-network information to the video streaming service. By doing so video streaming service can play at higher quality and reduce the amount of chunk prefetching. There are three main contributions in this paper:
\begin{itemize}
\item Design a scalable architecture to compute KPIs(key performance indicators) and expose them to video streaming providers on the fly.
\item Conduct a measurement study to show the predictability of user channel quality and cell loads. 
\item Design an algorithm to show that by using the KPIs we have a significant improvement in video streaming service.
\end{itemize}

This paper introduces a scalable architecture to compute the bandwidth and expose the KPIs to the service provider in \autoref{sec:Architecture} and then we conduct a preliminary measurement study of predictability of available bandwidth in \autoref{sec:prediction}. We propose our new optimization algorithm in \autoref{sec:optimization} and show our results in \autoref{sec:evaluation}. This paper concludes in \autoref{sec:conclusion}. 


