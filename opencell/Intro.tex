\section{Introduction}
\label{sec:intro}

Video streaming has been a dominant application traffic in cellular networks, it consists
up to 40\% of the peak load~\cite{LTENetwork, VideoMeasureatt}. Although
the video streaming has become one huge part of cellular networks, most video streaming algorithms are designed only for wireline networks~\cite{BBA, QDASH, Festive}. Unique characteristics from the cellular networks, such as bandwidth un-predictability over short period of
time due to handoff or fading effects, create a challenge for a satisfactory
video streaming service. 

Most of the current video streaming services are using DASH (\textbf{D}ynamic
\textbf{A}daptation \textbf{S}treaming over \textbf{H}TTP)~\cite{DASH} based
end-to-end approaches. It uses chunk-based downloading and adjusts its video
streaming bit rate based on the play progress and the monitored bandwidth. The
application stitches video chunks together to provide a continuous playback.
However most of the video streaming service are not tailored for cellular networks, and they suffer from a conservative bandwidth estimation and video quality oscillation due to a lack of knowledge for the network condition. 
Some low layer connection information such as signal information is hard to access from application layer without root access.
End-to-end mechanisms~\cite{Sprout, QDASH, Festive} have been proposed to estimate link throughput, however some aims to reduce the packet delay while trading off the throughput, while others are making conservative and rough estimation. Periodical downloading behavior~\cite{OnOff} makes the bandwidth prediction even harder.
%For TCP throughput, the application need constantly saturate the link for a better estimation and the estimation takes a long time to converge, which can be hardly used in video streaming case where downloading is periodical\cite{OnOff,Sprout}. 
%\xin{Not sure if this is correct. Can they be collected from end
%hosts? Cite some papers or explain why it's hard}, \kelvin{ how does this sound like now?}

On the other hand, a cellular network service provider has much better knowledge of the connection
conditions for the clients via monitoring, as the last mile has long been identified as the
bottleneck\cite{LASTMILE}. Cellular ISPs also know the number of
active users in each cell and proportional fair packet scheduling system, and thus can make more
precise predictions in terms of available bandwidth. In fact, many key statistics like radio information and the base station utilization can be easily and some are
already being collected within carriers. 

Given cellular carrier's advantages, there are some proposals\cite{Avis} for an in-network video streaming framework where the network controls video streaming algorithm. 
Nevertheless it also suffers from some limitations. First it
lacks the application knowledge such as client video buffer occupancy and may lead to re-buffering. 
Second different users may apply different subscription
policies w.r.t.video quality (Spotify already has low quality and high quality for free and premium users, video service may apply the same concept) and streaming service providers may
hesitate to share user information with ISPs. Users may also have data budget concerns. Last
but not least, in-network approaches may require the modification of packet scheduling system and this
is hard in practice as most carriers are using pre-installed
vendor-specific scheduler. 

To address the issue, we propose a joint optimization between the cellular network (ISP) and video streaming service (CP): ISP exposes its estimated bandwidth to CP, and CP adjusts its streaming service decision based on the estimated bandwidth. By doing so video streaming service can achieve higher quality and reduce the amount of chunk prefetching, and thus save data budget and power consumption. ISPs also benefit from less bursty traffic and avoid excessive queueing delay from flow pacing. There are three main contributions in this paper:
\begin{itemize}
\item Design a scalable architecture to compute  and expose them to video streaming providers on the fly.
\item Conduct a measurement study to show the predictability of user channel quality and cell loads. 
\item Design an algorithm to show that by using the KPIs we have a significant improvement in video streaming service.
\end{itemize}

This paper introduces a scalable architecture to compute the bandwidth and expose the KPIs to the service provider in \autoref{sec:Architecture} and then we conduct a preliminary measurement study of predictability of available bandwidth in \autoref{sec:prediction}. We propose our new optimization algorithm in \autoref{sec:optimization} and show our results in \autoref{sec:evaluation}. This paper concludes in \autoref{sec:conclusion}. 


=============== [rjana,gvijay] 

Increasing number of users are watching videos on their cellular
devices. Recent studies~\cite{cisco-vni} show that mobile video
accounts more than 50\% of the cellular traffic today and is expected
to grow fourteen fold by 2018. This growing popularity of video over
cellular networks has attracted significant attention with cellular
operators making huge investments to provide better coverage and
capacity, content providers adopting state-of-the-art delivery
mechanisms a like Dynamic Adaptation Streaming over HTTP
(DASH)~\cite{DASH} and HTTP Live Streaming (HLS)~\cite{HLS} and
researchers proposing mechanisms~\cite{HLS,Festive,BBA} that optimize
on metrics such as average video quality, number of interruptions, or
the stability of video quality.

Despite these novel delivery mechanisms and expensive network
investments, the quality of experience for users watching videos over
cellular networks remains low. A recent study~\cite{opera-stall} shows
that between 40\% and 73\% of all videos played on mobile networks
experience stalls. For every 60 seconds of video, users on 3G
experienced an average of 47 seconds of stall, while those on LTE
experienced 15 seconds stall~\cite{citrix-stall}. Worse yet, the
fraction of videos stalled increased with video quality with 10.5\% of
240p videos stalling while 45.7\% of 720p videos experienced a
stall. 

We believe that one of the main reasons for the low quality of
experience is the mis-match between information used by video
adaptation process and available network bandwidth. The adaptation
process on a video client has to decide on the video quality for each
chunk. To do so, it uses historical throughput as an estimate of the
client's current bandwidth. Based on its estimate of available
bandwidth, the client picks a certain video quality for the chunk. On
cellular networks, however, this estimate could be widely off. There
are many factors, including the user's signal strength, number of
users in the cell, the congestion in the cell, and user mobility that
can determine a user's bandwidth. As a result, a user's bandwidth can
vary widely even over short periods of time, rendering any historical
estimates inaccurate. Consequently, as we show in
Sec.~\ref{sec:background}, the client ends up either under- or
over-estimating its available bandwidth.

Our approach to tackling this issue is to move the task of bandwidth
estimation from the client to the cellular network. We envision
opening up the cellular network (e.g., through an API) to provide
hints to a client's player about the bandwidth available to that
client for a future time duration (typically few seconds). Our
motivation stems from the fact that the cellular operator has a global
view of all traffic that is flowing through their network. This
includes information such as users' signal strength (RSRP), wireless
channel quality (CQI, RSRQ), the number of users, users' mobility
patterns, etc. Further, the operator knows how the different
middleboxes in their network will affect this user's flow and know of
different priority treatments applicable to this user's
traffic. Having visibility of all competing flows inside the network
allows the cellular operator to {\em predict} every user's bandwidth
for a future time period. For example, the operator can predict, using
all this information, how a scheduler operating inside a base station
would rank the user and how much cellular resources it would allocate
for that user. This directly translates to the bandwidth that a user
would get.

In this paper, we first show that existing methods used by clients to
estimate cellular network bandwidth are indeed deficient. Then we
study if video quality of experience improves if we have accurate
network bandwidths. To do so, we use an oracle in an off-line mode to
provide predicted throughput. We show, using this oracle, that there
are significant improvements to be had by having accurate network
bandwidth information. This result also serves as an upper-bound on
the possible improvements. Next, assume that network information is
available to us, we show how we can predict the available bandwidth in
an online manner. Since existing video client algorithms are not aware
of our predictions, we design a client algorithm that uses this
information and compare its performance to the state-of-the-art
approaches (i.e., FESTIVE~\cite{Festive} and BBA~\cite{BBA}). Our
results show that our algorithm performs better than both these
approaches. Finally, we propose various mechanisms through which our
predicted bandwidth can be made available to video clients.

