\section{Introduction}
Video streaming has been a dominant application traffic in cellular networks, it consists
up to 40\% of the peak load~\cite{LTENetwork, VideoMeasureatt}. Although
the video streaming has become one huge part of cellular networks, most video streaming algorithms are designed only for wireline networks~\cite{BBA, QDASH, Festive}. Unique characteristics from the cellular networks, such as bandwidth un-predictability over short period of
time due to handoff or fading effects, create a challenge for a satisfactory
video streaming service. 

Most of the current video streaming services are using DASH (\textbf{D}ynamic
\textbf{A}daptation \textbf{S}treaming over \textbf{H}TTP)~\cite{DASH} based
end-to-end approaches. It uses chunk-based downloading and adjusts its video
streaming bit rate based on the play progress and the monitored bandwidth. The
application stitches video chunks together to provide a continuous playback.
However most of the video streaming service are not tailored for cellular networks, and they suffer from a conservative bandwidth estimation and video quality oscillation due to a lack of knowledge for the network condition. 
Some low layer connection information such as signal information is hard to access from application layer without root access.
End-to-end mechanisms~\cite{Sprout, QDASH, Festive} have been proposed to estimate link throughput, however some aims to reduce the packet delay while trading off the throughput, while others are making conservative and rough estimation. Periodical downloading behavior~\cite{OnOff} makes the bandwidth prediction even harder.
%For TCP throughput, the application need constantly saturate the link for a better estimation and the estimation takes a long time to converge, which can be hardly used in video streaming case where downloading is periodical\cite{OnOff,Sprout}. 
%\xin{Not sure if this is correct. Can they be collected from end
%hosts? Cite some papers or explain why it's hard}, \kelvin{ how does this sound like now?}

On the other hand, a cellular network service provider has much better knowledge of the connection
conditions for the clients via monitoring, as the last mile has long been identified as the
bottleneck\cite{LASTMILE}. Cellular ISPs also know the number of
active users in each cell and proportional fair packet scheduling system, and thus can make more
precise predictions in terms of available bandwidth. In fact, many key statistics like radio information and the base station utilization can be easily and some are
already being collected within carriers. 

Given cellular carrier's advantages, there are some proposals\cite{Avis} for an in-network video streaming framework where the network controls video streaming algorithm. 
Nevertheless it also suffers from some limitations. First it
lacks the application knowledge such as client video buffer occupancy and may lead to re-buffering. 
Second different users may apply different subscription
policies w.r.t.video quality (Spotify already has low quality and high quality for free and premium users, video service may apply the same concept) and streaming service providers may
hesitate to share user information with ISPs. Users may also have data budget concerns. Last
but not least, in-network approaches may require the modification of packet scheduling system and this
is hard in practice as most carriers are using pre-installed
vendor-specific scheduler. 

To address the issue, we propose a joint optimization between the cellular network (ISP) and video streaming service (CP): ISP exposes its estimated bandwidth to CP, and CP adjusts its streaming service decision based on the estimated bandwidth. By doing so video streaming service can achieve higher quality and reduce the amount of chunk prefetching, and thus save data budget and power consumption. ISPs also benefit from less bursty traffic and avoid excessive queueing delay from flow pacing. There are three main contributions in this paper:
\begin{itemize}
\item Design a scalable architecture to compute  and expose them to video streaming providers on the fly.
\item Conduct a measurement study to show the predictability of user channel quality and cell loads. 
\item Design an algorithm to show that by using the KPIs we have a significant improvement in video streaming service.
\end{itemize}

This paper introduces a scalable architecture to compute the bandwidth and expose the KPIs to the service provider in \autoref{sec:Architecture} and then we conduct a preliminary measurement study of predictability of available bandwidth in \autoref{sec:prediction}. We propose our new optimization algorithm in \autoref{sec:optimization} and show our results in \autoref{sec:evaluation}. This paper concludes in \autoref{sec:conclusion}. 


=============== [rjana,gvijay] 

Increasing number of users are watching videos on their cellular
devices. Recent studies~\cite{cisco-vni} show that mobile video
accounts more than 50\% of the cellular traffic today and is expected
to grow fourteen fold by 2018. This growing popularity of video over
cellular networks has resulted in significant attention by cellular
operators, content providers and the research community. 

Cellular operators, for example, are making huge investments to
provide better coverage and capacity in order to keep up with this
escalating demand. They are deploying the latest cellular technology
features at a rapid rate. Content providers are adopting
state-of-the-art delivery mechanisms a like Dynamic Adaptation
Streaming over HTTP (DASH)~\cite{DASH} and HTTP Live Streaming
(HLS)~\cite{HLS}. In these approaches, each video is segmented into chunks
(typically a few seconds) with each chunk encoded into multiple
quality levels (i.e., bitrate). The video client downloads one chunk
at a time, determining the quality of each chunk based on a number of
different considerations. Typical approaches~\cite{HLS,Festive,BBA}
try to determine the appropriate bitrate by optimizing for metrics
like the average video quality, number of interruptions, or the
stability of video quality. The decisions are usually based
on the duration of video buffered and an estimate of available network
bandwidth. 

These novel delivery mechanisms and expensive network investments,
however, have not translated to a high quality of experience for end
users. A recent study~\cite{opera-stall} shows that between 40\% and
73\% of all videos played on mobile networks experience stalls. For
every 60 seconds of video, users on 3G experienced an average of 47
seconds of stall, while those on LTE experienced 15 seconds
stall~\cite{citrix-stall}. Worse yet, the fraction of videos stalled
increased with video quality with 10.5\% of 240p videos stalling while
45.7\% of 720p videos experienced a stall. This implies that viewers
primarily get a low quality video with a lot of interruptions
resulting a low quality of experience with mobile vidoes.

We believe that one of the main reasons for the low quality of
experience is the mis-match between information used by video
adaptation process and available network bandwidth. The adaptation
process on a video client uses historical throughput to estimate its
current bandwidth. It then decides the quality of the next chunk based
on this estimate. On cellular networks, however, this estimate could
be widely off. There are many factors, including the user's signal
strength, number of users in the cell, the congestion in the cell, and
user mobility that can determine a user's bandwidth. As a result, a
user's bandwidth can vary widely even over short periods of time,
rendering any historical estimates inaccurate.

\textbf{VG: TODO FROM THIS POINT ON}
To this end, we propose opening up
the cellular network (OpenCell) to provide valuable hints (by means of
an API) to the player or the content server so that it is able to
provide the best quality encoding. OpenCell keeps account of all flows
on a per base station (e.g. LTE eNB) basis and tracks the measurement
reports received from the end user. The measurement reports provide
valuable indicators of the end users' signal strength (RSRP) and
wireless channel quality (CQI, RSRQ etc). This allows OpenCell to {\em
  predict} every user's channel quality for a future time period
(typically the next segment to be downloaded). An empirical
(data-driven) model is updated that maps channel quality to achievable
throughput. By making this prediction with a certain amount of
confidence, OpenCell is able to advise the player or the content
server which video quality to download next.

Why: The service provider has a global view of all traffic that are
flowing through the cellular network. In particular, having visibility
of all competing flows inside a base station allows the service
provider to allocate resources fairly and efficiently. The scheduler
that operates inside an eNB looks at the instantaneous measurement
reports and computes a metric to rank the user. Resources are then
proportionally allocated according to this rank. By predicting the
user's future throughput, OpenCell can decide on how to treat the
different flows (e.g. prioritize video and VOIP flows vs email and web
traffic). The operator also has visibility of the middle boxes (video
proxies) that exist between the content provider and the user. Video
flows are typically routed through these middle boxes to enforce
various policies (e.g. traffic shaping, admission control, etc.).

How: To understand the improvement obtained by using OpenCell, we
initially use an oracle in an off-line mode to provide predicted
throughputs. This allows us to quantify an upper bound the video QoE
metrics. Next, we design a video algorithm that performs better than
any of the state-of-the-art algorithms (e.g. FESTIVE, BBA,
etc.). Finally, we propose various mechanisms to realize OpenCell. We
leverage standard HTTP protocol headers to notify the player or the
content server on future segments to download (TCP headers can also be
used but we choose to implement using HTTP since it is more
accessible??). A proxy within the service provider's network can be
dedicated to collect all the relevant network level information and
perform the throughput prediction. Further shaping of traffic can also
be considered to prioritize concurrent flows inside an eNB.

Our contributions are
\begin{itemize}
\item Conduct a measurement study to show the predictability of users' channel quality and throughputs. 
\item Design an online buffer and channel-aware video algorithm to quantify the improvement in video QoE compared to state of the art.
\item Design a scalable solution and mechanisms to compute and publish relevant hints to the end user.
\end{itemize}   

This paper is structured as follows. Section 1 ....
