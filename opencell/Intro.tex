\section{Introduction}
\label{sec:intro}

\begin{comment}
Video streaming has been a dominant application traffic in cellular networks, as
it consists up to 40\% of the peak load~\cite{LTENetwork, VideoMeasureatt}.
But most video streaming algorithms are mainly designed for wireline
networks~\cite{BBA, QDASH, Festive}. Unique characteristics from the cellular
networks, such as bandwidth un-predictability over short period of time due to
handoff or fading effects, create a challenge for a satisfactory video streaming
service. 

Most of the current video streaming services are using DASH (\textbf{D}ynamic
\textbf{A}daptation \textbf{S}treaming over \textbf{H}TTP)~\cite{DASH} based
end-to-end approaches. It uses chunk-based downloading and adjusts its video
streaming bit rate based on the play progress and the monitored bandwidth. The
application stitches video chunks together to provide a continuous playback.
However most of the video streaming service are not tailored for cellular networks, and they suffer from a conservative bandwidth estimation and video quality oscillation due to a lack of knowledge for the network condition. 
Some low layer connection information such as signal information is hard to access from application layer without root access.
End-to-end mechanisms~\cite{Sprout, QDASH, Festive} have been proposed to estimate link throughput, however some aims to reduce the packet delay while trading off the throughput, while others are making conservative and rough estimation. Periodical downloading behavior~\cite{OnOff} makes the bandwidth prediction even harder.
%For TCP throughput, the application need constantly saturate the link for a better estimation and the estimation takes a long time to converge, which can be hardly used in video streaming case where downloading is periodical\cite{OnOff,Sprout}. 
%\xin{Not sure if this is correct. Can they be collected from end
%hosts? Cite some papers or explain why it's hard}, \kelvin{ how does this sound like now?}

On the other hand, a cellular network service provider has better knowledge of the connection
conditions for the clients via monitoring, as the last mile has long been identified as the
bottleneck\cite{LASTMILE}. Cellular ISPs also know the number of
active users in each cell and proportional fair packet scheduling system, and thus can make more
precise predictions in terms of available bandwidth. In fact, many key statistics like radio information and the base station utilization can be easily and some are
already being collected within carriers. 

Given cellular carrier's advantages, there are some proposals\cite{Avis} for an in-network video streaming framework where the network controls video streaming algorithm. 
Nevertheless it also suffers from some limitations. First it
lacks the application knowledge such as client video buffer occupancy and may lead to re-buffering. 
Second different users may apply different subscription
policies w.r.t.video quality (Spotify already has low quality and high quality for free and premium users, video service may apply the same concept) and streaming service providers may
hesitate to share user information with ISPs. Users may also have data budget concerns. Last
but not least, in-network approaches may require the modification of packet scheduling system and this
is hard in practice as most carriers are using pre-installed
vendor-specific scheduler. 

To address the issue, we propose a joint optimization between the cellular network (ISP) and video streaming service (CP): ISP exposes its estimated bandwidth to CP, and CP adjusts its streaming service decision based on the estimated bandwidth. By doing so video streaming service can achieve higher quality and reduce the amount of chunk prefetching, and thus save data budget and power consumption. ISPs also benefit from less bursty traffic and avoid excessive queueing delay from flow pacing. There are three main contributions in this paper:
\begin{itemize}
\item Design a scalable architecture to compute  and expose them to video streaming providers on the fly.
\item Conduct a measurement study to show the predictability of user channel quality and cell loads. 
\item Design an algorithm to show that by using the KPIs we have a significant improvement in video streaming service.
\end{itemize}

This paper introduces a scalable architecture to compute the bandwidth and expose the KPIs to the service provider in \autoref{sec:Architecture} and then we conduct a preliminary measurement study of predictability of available bandwidth in \autoref{sec:prediction}. We propose our new optimization algorithm in \autoref{sec:optimization} and show our results in \autoref{sec:evaluation}. This paper concludes in \autoref{sec:conclusion}. 



=============== [rjana,gvijay] 

Increasing number of users are watching videos on their cellular
devices. Recent studies~\cite{cisco-vni} show that mobile video
accounts more than 50\% of the cellular traffic today and is expected
to grow fourteen fold by 2018. This growing popularity of video over
cellular networks has attracted significant attention with cellular
operators making huge investments to provide better coverage and
capacity, content providers adopting state-of-the-art delivery
mechanisms like Dynamic Adaptation Streaming over HTTP
(DASH)~\cite{DASH} and HTTP Live Streaming (HLS)~\cite{HLS} and
researchers proposing mechanisms~\cite{HLS,Festive,BBA} that optimize
on metrics such as average video quality, number of interruptions, or
the stability of video quality.

Despite these novel delivery mechanisms and expensive network
investments, the quality of experience for users watching videos over
cellular networks remains low. A recent study~\cite{opera-stall} shows
that between 40\% and 73\% of all videos played on mobile networks
experience stalls. For every 60 seconds of video, users on 3G
experienced an average of 47 seconds of stall, while those on LTE
experienced 15 seconds stall~\cite{citrix-stall}. Worse yet, the
fraction of videos stalled increased with video quality with 10.5\% of
240p videos stalling while 45.7\% of 720p videos experienced a
stall. 

We believe that one of the main reasons for the low quality of
experience is the mis-match between information used by video
adaptation process and available network bandwidth. The adaptation
process on a video client has to decide on the video quality for each
chunk. To do so, it uses historical throughput as an estimate of the
client's current bandwidth. Based on its estimate of available
bandwidth, the client picks a certain video quality for the chunk. On
cellular networks, however, this estimate could be widely off. There
are many factors, including the user's signal strength, number of
users in the cell, the congestion in the cell, and user mobility that
can determine a user's bandwidth. As a result, a user's bandwidth can
vary widely even over short periods of time, rendering any historical
estimates inaccurate. Consequently, as we show in
Sec.~\ref{sec:background}, the client ends up either under- or
over-estimating its available bandwidth.

Our approach to tackling this issue is to move the task of bandwidth
estimation from the client to the cellular network. We envision
opening up the cellular network (e.g., through an API) to provide
hints to a client's player about the bandwidth available to that
client for a future time duration (typically few seconds). Our
motivation stems from the fact that the cellular operator has a global
view of all traffic that is flowing through their network. This
includes information such as users' signal strength (RSRP), wireless
channel quality (CQI, RSRQ), the number of users, users' mobility
patterns, etc. Further, the operator knows how the different
middleboxes in their network will affect this user's flow and know of
different priority treatments applicable to this user's
traffic. Having visibility of all competing flows inside the network
allows the cellular operator to {\em predict} every user's bandwidth
for a future time period. For example, the operator can predict, using
all this information, how a scheduler operating inside a base station
would rank the user and how much cellular resources it would allocate
for that user. This directly translates to the bandwidth that a user
would get.

In this paper, we first show that existing methods used by clients to
estimate cellular network bandwidth are indeed deficient. Then we
study if video quality of experience improves if we have accurate
network bandwidths. To do so, we use an oracle in an off-line mode to
provide predicted throughput. We show, using this oracle, that there
are significant improvements to be had by having accurate network
bandwidth information. This result also serves as an upper-bound on
the possible improvements. Next, assume that network information is
available to us, we show how we can predict the available bandwidth in
an online manner. Since existing video client algorithms are not aware
of our predictions, we design a client algorithm that uses this
information and compare its performance to the state-of-the-art
approaches (i.e., FESTIVE~\cite{Festive} and BBA~\cite{BBA}). Our
results show that our algorithm performs better than both these
approaches. Finally, we propose various mechanisms through which our
predicted bandwidth can be made available to video clients.



========== [emir, based on rjana,gvijay]

On-demand video streaming has rapidly become a dominant application in the Internet, accounting for 66\% of total consumer traffic in 2013, and projected to reach 79\% in 2018~\cite{cisco-vni}. In some cellular networks, over-the-top video has reached 40\% of the peak load~\cite{Erman:IMC:2011}. Such demand puts pressure on wireline and wireless Internet Service Providers (ISPs) to improve access link speeds and network coverage, in order to support high Quality of Experience (QoE) for their users.

One of the key drivers of video streaming success is the adoption of Adaptive-Bit-Rate (ABR) streaming over HTTP, which adjusts video quality in response to changing network conditions. The fundamental principle of ABR streaming is to match the required delivery rate of the selected video quality to the available end-to-end bandwidth between the content server and the client. However, even after years of deployment, many Internet streaming providers struggle to achieve high-quality reliable service over cellular networks, even over high-speed 4G cellular links. For example, it is reported that the fraction of stalled videos increases with video quality, with 10.5\% of 240p videos stalling while 45.7\% of 720p videos experiencing a stall~\cite{citrix-stall}. 

We believe that one of the main reasons for the low QoE is a mismatch between information used by a video adaptation process and available network bandwidth.  The adaptation process on a video client has to decide when and which video quality to request, based on some estimate of available bandwidth. This is not a trivial task in best-effort Internet, due to varying link capacities, congestion, and other factors. In cellular networks, this task is even more challenging, due to highly variable radio link, which is affected by signal strength, interference, noise, mobility, and other factors.

On the other hand, we recognize that cellular networks have a special feature that can offer a much better bandwidth prediction. The radio link in cellular networks is precisely scheduled, where radio metrics are combined to determine both the time and link rate available to each device. In addition, the architecture of the network allows an operator to have a global view of all devices, their link characteristics, and traffic demand. Our goal is to exploit this network information to improve cellular video QoE.

In this paper, we first demonstrate that traditional approach of bandwidth estimation using historical throughput is rendered invalid by the character of radio links, whose conditions can rapidly change at short time scale. We then postulate that in cellular networks, a more accurate prediction using network information can be achieved, leading to significant benefits in QoE. To this end, we develop Prediction-Based Adapation (PBA) class of algorithms, that are able to use the predicted available bandwidth for better and more stable video streaming. 

We first present an oracle-based offline PBA algorithm that establishes an upper bound on achievable video quality. Then, we develop an online PBA algorithm that uses several seconds of predicted bandwidth to achieve XX\% better quality, XX\% higher stability and superior outage tolerance over existing state-of-the-art proposals. After including critical stability and stall-free requirements, online PBA algorithm comes within 84\% of the upper bound on video quality. Finally, we discuss prediction methods and architectures for opening the cellular network and exposing the predictions via API.

\end{comment}
[emir v2]
On-demand video streaming has rapidly become a dominant application in the Internet, accounting for 66\% of total consumer traffic in 2013, and projected to reach 79\% in 2018~\cite{cisco-vni}. In some cellular networks, over-the-top video has reached 40\% of the peak load~\cite{Erman:IMC:2011}. Such demand puts pressure on wireline and wireless Internet Service Providers (ISPs) to improve access link speeds and network coverage, in order to support high Quality of Experience (QoE) for their users.

One of the key drivers of video streaming success is the adoption of Adaptive-Bit-Rate (ABR) streaming over HTTP. The fundamental principle of ABR streaming is to match the required delivery rate of the selected video quality to the available end-to-end bandwidth between the content server and the client. However, even after years of deployment, many Internet streaming providers struggle to achieve high-quality reliable service over cellular networks, even over high-speed 4G cellular links. For example, it is reported that the fraction of stalled videos increases with video quality, with 10.5\% of 240p videos stalling while 45.7\% of 720p videos experiencing a stall~\cite{citrix-stall}. 

The adaptation process on a video client has to decide when and which video quality to request. This decision is based on some estimate of available bandwidth, typically using historical throughput~\cite{QDASH,Festive} or pre-fetch buffer level~\cite{BBA}. We believe that the key reason for low QoE is a mismatch between information used by the existing adaptation algorithms and the actual available network bandwidth. The best-effort Internet poses a non-trivial task to such adaptation logic, due to varying link capacities, congestion, and other factors. In cellular networks, this task is even more challenging, 
%due to a highly variable radio link at sub-second level, which is affected by signal strength, interference, noise, mobility, and other factors.
since factors like signal strength, interference, noise, and mobility cause link variations at sub-second level.

In this paper, we focus on the following question: \emph{If accurate bandwidth prediction were possible in a cellular network, would this substantially improve ABR video QoE}? We consider cellular networks for two reasons. First, it is a very challenging environment with high link variability. Second, traffic over a typical bottleneck, the radio link, is carefully scheduled. A base station tracks multiple network metrics (routinely used in the scheduling process) that can be used to predict available bandwidth in near future (several seconds). In addition, the architecture of the network allows an operator to have a global view of all devices, their link characteristics, and traffic demand. Our goal is to exploit this network information and expose it to the service providers.

To this end, we first demonstrate that traditional approach of bandwidth estimation using historical throughput~\cite{Festive} is rendered invalid by the variable character of radio links. 
%Then, we postulate that in cellular networks, a more accurate prediction using network information can be achieved, leading to significant benefits in QoE. 
Since state-of-the-art adaptation algorithms~\cite{BBA,QDASH,Festive} are not designed to use network-provided prediction, we develop a class of Prediction-Based Adapation (PBA) algorithms, that are able to use the predicted available bandwidth for improving video QoE. 

We first present an oracle-based offline PBA algorithm that establishes an upper bound on achievable video quality. Then, we develop an online PBA algorithm that uses several seconds of predicted bandwidth to achieve 26\% better quality, 170\% higher stability and superior outage tolerance over existing state-of-the-art proposals. After including critical stability and stall-free requirements, online PBA algorithm comes within 84\% of the upper bound on video quality. 

After establishing that prediction brings significant gain, we discuss prediction methods and architectures for opening the cellular network and exposing the predictions via API.





